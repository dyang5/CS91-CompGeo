\documentclass[11pt]{article}
\usepackage{styletemplate} % Required for inserting images

\begin{document}

\hrule
\begin{center}
    \textbf{CS91T: Computational Geometry}\hfill \textbf{Fall 2023}\newline

    {\Large Circularity of Convex Partitions}

    David Yang
\end{center}

\hrule

\vspace{1em}

\section{Problem Setup}
In this project, I explored [Open Problem 59], the problem of partitioning a polygon into pieces, each of which is as
``circular'' as possible. This problem was first introduced by [] and was expanded upon by Mirela Damian and Joseph O'Rourke. \\


One notion of the circularity of a given polygon is its \textit{aspect ratio} -- the more ``circular'' a partition, the closer its aspect ratio is to $1$.

\begin{definition}[Aspect Ratio]
The \textbf{aspect ratio} of a polygon is the ratio of the diameters of the smallest circumscribing circle to the largest inscribed disk. \\

\textit{The distinction between circle and disk is made to emphasize that the circumcircles can intersect, whereas the disks cannot.}
\end{definition}

In Open Problem 59 and this project, we focus on minimizing the aspect ratio across a set of convex partitions of a regular polygon.

\begin{definition}[Partition]
A \textbf{partition} of a polygon $\mathcal{P}$ is a collection of polygonal pieces $P_1, P_2 \dots $ such that $\mathcal{P} = \cup P_i$ and no pair of pieces share an interior point. 
\end{definition}

In our problem, we add the restriction that our partition of a regular polygon $P$ must be convex, meaning that each of its polygonal pieces are themselves convex. Finally, we study the 
circularity, an extension of aspect ratio, of our convex partition.

\begin{definition}[Circularity]
The \textbf{circularity} of a convex partition of a polygon $\mathcal{P}$ is the maximum aspect ratio, across all polygonal pieces of the partition.
\end{definition}

In summary, the problem is as follows:
\begin{goal*}Find the minimum circularity across all convex partitions of a regular polygon.
\end{goal*}

\section{Related Problems}

Calculating the circularity is difficult. 

\section{Approach and Methodology}



\end{document}
