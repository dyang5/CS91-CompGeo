\documentclass[11pt]{article}
\usepackage{styletemplate} % Required for inserting images

\begin{document}

\hrule
\begin{center}
    \textbf{CS91T: Computational Geometry}\hfill \textbf{Fall 2023}\newline

    {\Large Circularity of Convex Partitions}

    David Yang
\end{center}

\hrule

\vspace{1em}

\section{Problem Setup}
In this project, I explored TOPP Open Problem 59 \cite{topp}, the problem of partitioning a polygon into pieces, each of which is as
``circular'' as possible. This problem was first studied by Mirela Damian and Joseph O'Rourke \cite{damian_rourke}. \\


One notion of the circularity of a given polygon is its \textit{aspect ratio} -- the more ``circular'' a partition, the closer its aspect ratio is to $1$.

\begin{definition}[Aspect Ratio \& Circularity]
The \textbf{aspect ratio}, or \textbf{circularity}, of a polygon is the ratio of the diameters of the smallest circumscribing circle to the largest inscribed disk. \\

\textit{The distinction between circle and disk is made to emphasize that the circumcircles can intersect, whereas the disks cannot.}
\end{definition}

In \cite{topp} and this project, we focus on minimizing the aspect ratio across a set of convex partitions of a regular polygon.

\begin{definition}[Partition]
A \textbf{partition} of a polygon $\mathcal{P}$ is a collection of polygonal pieces $P_1, P_2 \dots $ such that $\mathcal{P} = \cup P_i$ and no pair of pieces share an interior point. 
\end{definition}

In our problem, we add the restriction that our partition of a regular polygon $P$ must be convex, meaning that each of its polygonal pieces are themselves convex. \\

In summary, the problem is as follows:
\begin{goal*}Find convex partitions of regular polygons that minimize the maximum circularity across all pieces in the partition.
\end{goal*}

\section{Related Problems}

Calculating the circularity is difficult. 

Related to Smallest Enclosing circle

and Largest Empty Circle (but LEC finds largest circle not containing vertices), while we care about circle not hitting any edges.
This is a problem of finding the Voronoi diagram of edges (discussed in de Berg), or the medial axis of the given polygon. However, there were no implementations online so I used a brute force algorithm.
\section{Approach and Methodology}

\newpage

\section{Future Work: Extensions}

We focus on square, but can extend to any polygon. Idea: place points equally spaced on circle, to form bounding polygon. 

\begin{thebibliography}{9}
    \bibitem{topp}
    Erik D. Demaine, Joseph S. B. Mitchell, Joseph O'Rourke. \textit{The Open Problems Project}. 
    Originated from Computational Geometry Column 42. Internat. J. Comput. Geom. Appl., 11(5):573-582, 2001. Also in SIGACT News 32(3):63-72 (2001), Issue 120.

    \bibitem{damian_rourke}
    Mirela Damian and Joseph O'Rourke. \textit{Partitioning Regular Polygons into Circular Pieces I: Convex Partitions.} In Proc. 15th Canad. Conf. Comput. Geom., pages 43-46, 2003. arXiv:cs.CG/030402.    


\end{thebibliography}

\end{document}
